\documentclass[twoside,11pt]{article}
\usepackage{amssymb}
\title{IEOR 4500 Application Programming for FE \\
  \large Assignment 2
} % DON'T CHANGE THIS

\author{Anurag Dutt, Shrey Goel, Jatindeep Singh, Vinayak Shinde, Aditya Zalte}    % YOUR NAME GOES HERE
%\studmail{}% YOUR UNI GOES HERE
%\hwNo{1}                   % THE HOMEWORK NUMBER GOES HERE
%\collab{Group 19}   % THE UNI'S OF STUDENTS YOU DISCUSSED WITH

% Uncomment the next line if you want to use \includegraphics.
\usepackage{graphicx}
\usepackage{amsmath}
\begin{document}
\maketitle
This file includes te write-up and explanations for any of the problems in the aforementioned assignment

\section*{Problem 2}
% YOUR SOLUTION GOES HERE
In the second part we transform the Q matrix as follows:-
\begin{equation}
  Q = V^TFV
\end{equation}
F is the diagonal matrix of the eigen values.
V is 2x2 matrix because the first two assets with highest eigenvalues
explain over 95\% variability of the portfolio. (Results from a PCA analysis)
The optimization problem that we are trying to solve is:-

\begin{equation}
  F(x) = min.\; \lambda x^TQx - \mu^Tx
\end{equation}
where the constraints are:-
\begin{equation}
  1^Tx = 1
\end{equation}
\begin{equation}
  Ix \geqslant L
\end{equation}
\begin{equation}
  (-I)x \geqslant -U
\end{equation}
where L are the lower bounds vector and U is the upper bound vector.\\
Substituting Q as shown in equation 1, the minimization problem
changes to:
\begin{equation}
  F(x) = min.\; \lambda x^TV^TFVx - \mu^Tx
\end{equation}
where the constraints are:-
\begin{equation}
  1^Tx = 1
\end{equation}
\begin{equation}
  Ix \geqslant L
\end{equation}
\begin{equation}
  (-I)x \geqslant -U
\end{equation}
Taking the substitution:-
\begin{equation}
  Vx = y
\end{equation}
We do this substitution because our current function for solving the
quadratic program accepts input in terms of $A^TBA$ and using the
property of V:
\begin{equation}
  V^T = V^{-1}
\end{equation}

Hence the optimization problem changes to:-
\begin{equation}
  F(x) = min.\; \lambda y^TFy - \mu^TV^Ty
\end{equation}
where the constraints are:-
\begin{equation}
  1^TV^Ty = 1
\end{equation}
\begin{equation}
  V^Ty \geqslant L
\end{equation}
\begin{equation}
  -V^Ty \geqslant -U
\end{equation}

Again for verifying the constraint:
\begin{equation}
  1^TV^Ty = 1
\end{equation}
Our code accepts a vector of portfolio weights. Hence we introduce a
matrix D such that:
\begin{equation}
  Dy = z
\end{equation}
or
\begin{equation}
  y = D^{-1}z
\end{equation}
For example in our case the constraint transforms to:-

\[
1 =
  \begin{bmatrix}
    -0.999 & 0.8523    
  \end{bmatrix}
  \begin{bmatrix}
    y_1 \\
    y_2    
  \end{bmatrix}
\]
\begin{equation}
  z_1 = -0.999y_1
\end{equation}
\begin{equation}
  z_2 = -0.8523y_2
\end{equation}
And hence we get the vector of z such that:-
\[
  \begin{bmatrix}
    z_1 \\
    z_2    
  \end{bmatrix}
 =
  \begin{bmatrix}
    -0.999 & 0 \\
    0 & 0.8523
    
  \end{bmatrix}
  \begin{bmatrix}
    y_1 \\
    y_2    
  \end{bmatrix}
\]
Hence the D matrix in our case is:
\[
D =
  \begin{bmatrix}
    -0.999 & 0 \\
    0 & 0.8523
    
  \end{bmatrix}
\]

Hence the optimization problem finally changes to:-
\begin{equation}
  F(x) = min.\; \lambda z^TD^{-1}FD^{-1}z - \mu^TV^TD^{-1}z
\end{equation}
where the constraints are:-
\begin{equation}
  1^Z = 1
\end{equation}
\begin{equation}
  Z \geqslant DVL
\end{equation}
\begin{equation}
  -Z \geqslant -DVU
\end{equation}


\end{document} 
